\documentclass[a4paper,10pt]{article}

\usepackage[brazilian]{babel}
\usepackage[utf8]{inputenc}
\usepackage[T1]{fontenc}
\usepackage{titlesec}

\titleformat{\section}
  {\normalfont\scshape\bfseries}{\thesection}{1em}{}
\titleformat{\paragraph}
  {\normalfont}{\theparagraph}{1em}{}

\title{\textbf{Aprendizado Automático de Sum-Product Networks (SPN)}}

\begin{document}
\date{}
\author{}
\vspace*{-40pt}
{\let\newpage\relax\maketitle}

Projeto de MAC0215 (Atividade Curricular em Pesquisa)

Aluno: Renato Lui Geh (Bacharelado em Ciência da Computação)

Orientador: Denis Deratani Mauá
\section{Introdução}

Sum-Product Networks são uma nova classe de modelos probabilísticos cuja inferência é sempre
tratável.


\end{document}
