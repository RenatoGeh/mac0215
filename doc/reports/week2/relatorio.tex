\documentclass[a4paper,10pt]{article}

\usepackage[brazilian]{babel}
\usepackage[utf8]{inputenc}
\usepackage[T1]{fontenc}
\usepackage{titlesec}
\usepackage{graphicx}
\usepackage{mathtools}
\usepackage{amsthm}
\usepackage[top=1.0in,bottom=1.0in]{geometry}
\usepackage{hyperref}
\usepackage[singlelinecheck=false]{caption}
\usepackage[backend=biber,url=true,doi=true,eprint=false]{biblatex}

\addbibresource{../common/references.bib}

\newcommand\blfootnote[1]{%
  \begingroup
  \renewcommand\thefootnote{}\footnote{#1}%
  \addtocounter{footnote}{-1}%
  \endgroup
}

\newcommand\defeq{\mathrel{\overset{\makebox[0pt]{\mbox{\normalfont\tiny\sffamily def}}}{=}}}

\titleformat{\section}
  {\normalfont\scshape\bfseries}{\thesection}{1em}{}
\titleformat{\subsection}
  {\normalfont\scshape\bfseries}{\thesubsection}{1em}{}
\titleformat{\paragraph}
  {\normalfont}{\theparagraph}{1em}{}
\titleformat{\subparagraph}
  {\normalfont}{\thesubparagraph}{1em}{}

\captionsetup[table]{labelsep=space}

\theoremstyle{plain}

\newtheorem*{spn-def}{Definição}
\newtheorem*{spn-thm}{Teorema}

\title{\textbf{Modeling and Reasoning with Bayesian Networks: Inference by Variable Elimination 6.1-6.5}}

\begin{document}
\date{}
\author{}
\vspace*{-40pt}
{\let\newpage\relax\maketitle}

Relatório semana 2 - MAC0215 (Atividade Curricular em Pesquisa)

Aluno: Renato Lui Geh (Bacharelado em Ciência da Computação)

Orientador: Denis Deratani Mauá

\section{Atividades realizadas na semana}

\paragraph{
  Durante a semana foram lidos os seguintes tópicos do livro \textit{Modeling and Reasoning with
Bayesian Networks}\cite{bayes-net-darwiche}:
}

\begin{description}
  \item[6] - Inference by Variable Elimination
  \begin{description}
    \item[6.1] - Introduction
    \item[6.2] - The Process of Elimination
    \item[6.3] - Factors
    \item[6.4] - Elimination as a Basis for Inference
    \item[6.5] - Computing Prior Marginals
  \end{description}
\end{description}

\section{Definição das atividades}

\paragraph{
  Foi estudado o processo de eliminação de uma variável em uma Rede Bayesiana, simplificando a rede
para computarmos inferência. Como tópicos importantes temos a definição de fatores 
(\textit{factors}), as operações de soma (\textit{summing out}) e multiplicação de factors e como
chegarmos em inferência e \textit{prior marginals} a partir de eliminação de variáveis.
}

\paragraph{
  Esta seção será dividida em subseções para cada tópico que citamos no parágrafo anterior com a
inclusão de um tópico de introdução para assuntos que foram pesquisados fora dos subcapítulos lidos
e que não estão presentes nos outros relatórios. Abaixo está a lista de tópicos deste relatório.
}

\begin{enumerate}
  \item Introdução
  \item Factors
  \item Summing out factors
  \item Multiplying factors
  \item Eliminação e propriedades
  \item Computando prior marginals
\end{enumerate}

\subsection{Introdução}

\newpage

\printbibliography

\end{document}
