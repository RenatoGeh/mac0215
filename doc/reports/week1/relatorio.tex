\documentclass[a4paper,10pt]{article}

\usepackage[brazilian]{babel}
\usepackage[utf8]{inputenc}
\usepackage[T1]{fontenc}
\usepackage{titlesec}
\usepackage{graphicx}
\usepackage{mathtools}
\usepackage{amsthm}
\usepackage[top=1.0in,bottom=1.0in]{geometry}
\usepackage{hyperref}
\usepackage[singlelinecheck=false]{caption}
\usepackage[backend=biber,url=true,doi=true,eprint=false]{biblatex}

\addbibresource{../common/references.bib}

\newcommand\blfootnote[1]{%
  \begingroup
  \renewcommand\thefootnote{}\footnote{#1}%
  \addtocounter{footnote}{-1}%
  \endgroup
}

\titleformat{\section}
  {\normalfont\scshape\bfseries}{\thesection}{1em}{}
\titleformat{\subsection}
  {\normalfont\scshape\bfseries}{\thesubsection}{1em}{}
\titleformat{\paragraph}
  {\normalfont}{\theparagraph}{1em}{}
\titleformat{\subparagraph}
  {\normalfont}{\thesubparagraph}{1em}{}

\captionsetup[table]{labelsep=space}

\theoremstyle{plain}

\newtheorem*{spn-def}{Definição}

\title{\textbf{Modeling and Reasoning with Bayesian Networks: Compiling Bayesian Networks}}

\begin{document}
\date{}
\author{}
\vspace*{-40pt}
{\let\newpage\relax\maketitle}

\newpage

\section{Atividades realizadas na semana}

\paragraph{
  Durante a semana, foram lidos os seguintes tópicos do livro \textit{Modeling and Reasoning with
Bayesian Networks}\cite{bayes-net-darwiche}:
}

\begin{description}
  \item[12] - Compiling Bayesian Networks
  \begin{description}
    \item[12.1] - Introduction
    \item[12.2] - Circuit semantics
    \item[12.3] - Circuit propagation
    \begin{description}
      \item[12.3.1] - Evaluation and differentiation passes
    \end{description}
  \end{description}
\end{description}

\section{Definição das atividades}

\paragraph{
  Foram estudados o processo de se compilar Redes Bayesianas em circuitos aritméticos, algumas 
notações usadas em Redes Bayesianas, a definição de uma \textit{network polynomial}, algumas 
propriedades de Redes Bayesianas e diferenciação de uma rede a partir de uma evidência.
}

\paragraph{
  Essa seção será dividida em subseções para cada subtópico citado na seção anterior.
}

\subsection{Introduction}

\paragraph{
  Neste tópico foram estudados algumas notações usadas em PGMs 
}

\section{Referências}

\blfootnote{Os artigos listados na referência acima podem ser encontrados em: 
  \url{http://www.ime.usp.br/~renatolg/mac0215/articles/}}

\printbibliography[title={Artigos},type=article]
\printbibliography[title={Websites},type=misc]

\end{document}
