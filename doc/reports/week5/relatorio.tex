\documentclass[a4paper,10pt]{article}

\usepackage[brazilian]{babel}
\usepackage[utf8]{inputenc}
\usepackage[T1]{fontenc}
\usepackage{titlesec}
\usepackage{graphicx}
\usepackage{mathtools}
\usepackage{amsthm}
\usepackage[top=1.0in,bottom=1.0in]{geometry}
\usepackage{hyperref}
\usepackage[singlelinecheck=false]{caption}
\usepackage[backend=biber,url=true,doi=true,eprint=false]{biblatex}
\usepackage{enumitem}
\usepackage[x11names, rgb]{xcolor}
\usepackage{tikz}
\usetikzlibrary{snakes,arrows,shapes}

\addbibresource{../common/references.bib}

\newcommand\blfootnote[1]{%
  \begingroup
  \renewcommand\thefootnote{}\footnote{#1}%
  \addtocounter{footnote}{-1}%
  \endgroup
}

\newcommand\defeq{\mathrel{\overset{\makebox[0pt]{\mbox{\normalfont\tiny\sffamily def}}}{=}}}

\titleformat{\section}
  {\normalfont\scshape\bfseries}{\thesection}{1em}{}
\titleformat{\subsection}
  {\normalfont\scshape\bfseries}{\thesubsection}{1em}{}
\titleformat{\paragraph}
  {\normalfont}{\theparagraph}{1em}{}
\titleformat{\subparagraph}
  {\normalfont}{\thesubparagraph}{1em}{}

\captionsetup[table]{labelsep=space}

\theoremstyle{plain}

\newtheorem*{spn-def}{Definição}
\newtheorem*{spn-thm}{Teorema}

\title{\textbf{Modeling and Reasoning with Bayesian Networks: Inference by Variable Elimination 6.6-6.9}}

\begin{document}
\date{}
\author{}
\vspace*{-40pt}
{\let\newpage\relax\maketitle}

Relatório semana 5 - MAC0215 (Atividade Curricular em Pesquisa)

Aluno: Renato Lui Geh (Bacharelado em Ciência da Computação)

Orientador: Denis Deratani Mauá

\section{Atividades realizadas na semana}

\paragraph{
  Durante a semana foram lidos os seguintes tópicos do livro \textit{Modeling and Reasoning with
  Bayesian Networks}\cite{bayes-net-darwiche}:
}

\begin{description}
  \item[6] - Inference by Variable Elimination
  \begin{description}
    \item[6.6] - Choosing an Elimination Order
    \item[6.7] - Computing Posterior Marginals
    \item[6.8] - Network Structure and Complexity
    \item[6.9] - Query Structure and Complexity
  \end{description}
\end{description}

\section{Definição das atividades}

\paragraph{
  Foram estudados como encontrar a melhor ordem de eliminação de variáveis\cite{report-2}, 
  definimos o que são grafos de interação de Redes Bayesianas, como calcularmos marginais
  posteriores (\textit{posterior marginals}), algumas propriedades da estrutura de uma rede e como 
  podar (\textit{prune}) nós e arestas de uma Rede Bayesiana.
}

\paragraph{
  Esta seção será dividida em subseções onde veremos os itens enumerados no parágrafo acima. A cada
  tópico que não tenha sido apresentado ainda vamos abordar a teoria por trás e introduzir o 
  assunto. Separaremos tais explicações em uma subsubseção equivalente. Abaixo está a lista de 
  tópicos que iremos abordar.
}

\begin{enumerate}
  \item Escolhendo uma ordem de eliminação
    \begin{enumerate}[label*=\arabic*.]
      \item Subgraphs, induced subgraphs, cliques e spanning subgraphs
      \item Grafo de interação
    \end{enumerate}
  \item Computando posterior marginals
    \begin{enumerate}[label*=\arabic*.]
      \item Posterior marginals e joint marginals
      \item Eliminação de variáveis e posterior marginals
    \end{enumerate}
  \item Estrutura e complexidade da rede
    \begin{enumerate}[label*=\arabic*.]
      \item Treewidth
      \item Tree networks, polytrees e multiply connected
    \end{enumerate}
  \item Pruning
    \begin{enumerate}[label*=\arabic*.]
      \item Pruning nodes
      \item Pruning edges
      \item Network pruning
    \end{enumerate}
  \item Apêndice
    \begin{enumerate}[label*=\arabic*.]
      \item NP-hard e NP-complete
    \end{enumerate}
\end{enumerate}

\subsection{Escolhendo uma ordem de eliminação}

\paragraph{
  Como vimos no relatório anterior\cite{report-2}, queremos criar factors que sejam os menores 
  possíveis. A ordem de eliminação que tiver menor $w$ será a melhor. Dizemos que o $w$ de uma 
  ordem de eliminação é o maior $width$ de todos os factors da ordem de eliminação. 
}

\paragraph{
  Antes de acharmos a melhor ordem de eliminação precisamos saber como calcular o $w$ de uma ordem.
  Um jeito ingênuo de acharmos é, a cada interação $i$ de uma multiplicação e soma, acharmos o 
  $w_i$ com respeito a $\pi(i)$. No final retornamos o maior $w_i$. 
}

\begin{table}[h]
  \begin{center}
    \begin{tabular}{*{4}{l|} l}
      i & $\pi(i)$ & $\mathcal{S}$ & $f_i$ & $w$ \\
      \hline
      & & $\Theta_A \Theta_{B|A} \Theta_{C|A} \Theta_{D|BC} \Theta_{E|C}$ & & \\
      & & & & \\
      1 & B & $\Theta_A \Theta_{C|A} \Theta_{E|C} f_1(A, C, D)$ & $f_1 = \sum_B \Theta_{B|A} \Theta_{D|BC}$ & 3 \\
      2 & C & $\Theta_A f_2(A, D, E)$ & $f_2 = \sum_C \Theta_{C|A} \Theta_{E|C} f_1(A, C, D)$ & 3 \\
      3 & A & $f_3(D, E)$ & $f_3 = \sum_A \Theta_A f_2(A, D, E)$ & 2 \\
      4 & D & $f_4(E)$ & $f_4 = \sum_D f_3(D, E)$ & 1 \\
    \end{tabular}
  \end{center}
\end{table}

\paragraph{
  Na tabela acima a segunda coluna indica a variável $\pi(i)$ que queremos eliminar. A terceira 
  coluna é o resultado da substituição de $\pi(i)$ pelo factor resultante na quarta coluna, onde
  multiplicamos e somamos as CPTs onde $\pi(i)$ está presente. A última coluna é o $w_i$ de cada
  eliminação. Pode-se ver que o $w$ dessa ordem é 3, já que é o maior $w$ entre todas as iterações.
  Note que não precisamos realmente computar a eliminação das variáveis. De fato não fazemos tais
  eliminações, mas a cada $\pi(i)$ que aparecem nos factors $f(\textbf{X}_k)$ substituímos tais
  factors por um novo factor sob as variáveis $\cup_k \textbf{X}_k \setminus {\pi(i)}$.
}

\paragraph{
  Um outro jeito de se achar a $width$ de uma ordem de eliminação é por meio de um grafo não 
  direcionado que representa as interações das CPTs de uma Rede Bayesiana. Antes de definirmos
  esse grafo vamos introduzir alguns conceitos fundamentais sobre grafos.
}

\subsubsection{Subgraphs, induced subgraphs, cliques e spanning subgraphs}

\paragraph{
  Vamos definir $V(G)$ como o conjunto de vértices do grafo G. Similarmente, $E(G)$ é o conjunto de
  arestas do grafo $G$. Dizemos que um grafo é não direcionado se para cada vértice $x, y$ em $G$,
  tanto $xy$ quanto $yx$ são arestas válidas em $G$.
}

\paragraph{
  Primeiro vamos ver a definição de subgrafo:
}

\begin{spn-def} Um grafo $H$ é um subgrafo de $G$ se $V(H)$ é um subconjunto de $V(G)$ e $E(H)$ é
  subconjunto de $E(G)$. Dizemos que $G$ é o supergrafo de $H$ se $H$ é subgrafo de $G$.
\end{spn-def}

\paragraph{
  Vamos agora definir o que é um grafo induzido:
}

\begin{spn-def} Um subgrafo $H$ de um grafo $G$ é dito induzido se para cada par de vértices $x, y$
  em $H$, $xy$ é uma aresta de $H$ se e somente se $xy$ é uma aresta em $G$. Ou seja, $H$ é 
  um subgrafo induzido de $G$ se as arestas em $H$ são exatamente as arestas que aparecem em $G$ 
  sob o mesmo conjunto de vértices. Se $V(H)$ é um subconjunto $S$ de $V(G)$, então $H$ pode ser
  escrito como $G[S]$ e é dito ser induzido por $S$.
\end{spn-def}

\paragraph{
  Agora iremos definir o que é a completude de um grafo:
}

\begin{spn-def} Dizemos que um grafo $G$ é completo se todo par de vértices distintos em $G$ é 
  conectado por uma aresta única.
\end{spn-def}

\paragraph{
  Definimos agora o que é um clique:
}

\begin{spn-def} Um clique é um subconjunto de vértices de um grafo não direcionado tal que o 
  subgrafo induzido dele seja completo. Ou seja, todo par distinto de vértices em um clique é 
  adjacente.
\end{spn-def}

\paragraph{
  Vamos também definir o que é um \textit{spanning subgraph}:
}

\begin{spn-def} Dizemos que um subgrafo $H$ cobre (\textit{spans}) um grafo $G$ se ele tem mesmo
  conjunto de vértices que $G$. $H$ é então dito o \textit{spanning subgraph} de $G$.
\end{spn-def}

\subsubsection{Grafo de interação}

\paragraph{
  Agora que temos uma noção básica sobre grafos podemos definir o grafo de interação.
}

\begin{spn-def} Seja $f_1,...,f_n$ o conjunto de factors. O \textit{grafo de interação} $G$ desses
  factors é um grafo não direcionado construído de tal forma que os nós de $G$ são as variáveis que
  aparecem nos factors $f_1,...,f_n$. Há uma aresta entre duas variáveis em $G$ se e somente se 
  essas variáveis aparecem no mesmo factor.
\end{spn-def}

\paragraph{
  Em outras palavras, as variáveis $\textbf{X}_i$ de cada factor $f_i$ formam um clique em $G$, ou
  seja, as variáveis emparelhadas são adjacentes.
}

\begin{table}[h]
  \begin{center}
    \begin{tabular}{l | c}
      $\mathcal{S}_1: \Theta_A \Theta_{B|A} \Theta_{C|A} \Theta_{D|BC} \Theta_{E|C}$ & 
\begin{tikzpicture}[>=latex,line join=bevel,]
  \pgfsetlinewidth{1bp}
%%
\pgfsetcolor{black}
  % Edge: B -- C
  \draw [] (36.112bp,63.837bp) .. controls (52.183bp,63.837bp) and (75.539bp,63.837bp)  .. (91.599bp,63.837bp);
  % Edge: B -- A
  \draw [] (31.012bp,76.849bp) .. controls (37.259bp,83.095bp) and (44.721bp,90.557bp)  .. (50.95bp,96.787bp);
  % Edge: C -- E
  \draw [] (127.96bp,63.837bp) .. controls (133.59bp,63.837bp) and (139.78bp,63.837bp)  .. (145.41bp,63.837bp);
  % Edge: C -- D
  \draw [] (96.661bp,50.824bp) .. controls (90.415bp,44.578bp) and (82.953bp,37.116bp)  .. (76.723bp,30.886bp);
  % Edge: A -- C
  \draw [] (76.849bp,96.661bp) .. controls (83.095bp,90.415bp) and (90.557bp,82.953bp)  .. (96.787bp,76.723bp);
  % Edge: D -- B
  \draw [] (50.824bp,31.012bp) .. controls (44.578bp,37.259bp) and (37.116bp,44.721bp)  .. (30.886bp,50.95bp);
  % Node: A
\begin{scope}
  \definecolor{strokecol}{rgb}{0.0,0.0,0.0};
  \pgfsetstrokecolor{strokecol}
  \draw (63.837bp,109.67bp) node {$A$};
\end{scope}
  % Node: C
\begin{scope}
  \definecolor{strokecol}{rgb}{0.0,0.0,0.0};
  \pgfsetstrokecolor{strokecol}
  \draw (109.67bp,63.837bp) node {$C$};
\end{scope}
  % Node: B
\begin{scope}
  \definecolor{strokecol}{rgb}{0.0,0.0,0.0};
  \pgfsetstrokecolor{strokecol}
  \draw (18.0bp,63.837bp) node {$B$};
\end{scope}
  % Node: E
\begin{scope}
  \definecolor{strokecol}{rgb}{0.0,0.0,0.0};
  \pgfsetstrokecolor{strokecol}
  \draw (163.67bp,63.837bp) node {$E$};
\end{scope}
  % Node: D
\begin{scope}
  \definecolor{strokecol}{rgb}{0.0,0.0,0.0};
  \pgfsetstrokecolor{strokecol}
  \draw (63.837bp,18.0bp) node {$D$};
\end{scope}
%
\end{tikzpicture}

 \\
      \hline
      $\mathcal{S}_2: \Theta_A \Theta_{C|A} \Theta_{E|C} f_1(A, C, D)$ & 
\begin{tikzpicture}[>=latex,line join=bevel,]
  \pgfsetlinewidth{1bp}
%%
\pgfsetcolor{black}
  % Edge: C -- D
  \draw [] (64.122bp,39.432bp) .. controls (60.828bp,37.531bp) and (57.464bp,35.588bp)  .. (54.172bp,33.688bp);
  % Edge: A -- C
  \draw [] (54.336bp,76.647bp) .. controls (57.63bp,74.745bp) and (60.994bp,72.803bp)  .. (64.285bp,70.903bp);
  % Edge: D -- A
  \draw [] (27.0bp,36.017bp) .. controls (27.0bp,47.633bp) and (27.0bp,62.686bp)  .. (27.0bp,74.32bp);
  % Edge: C -- E
  \draw [] (118.54bp,55.215bp) .. controls (121.48bp,55.215bp) and (124.46bp,55.215bp)  .. (127.4bp,55.215bp);
  % Node: A
\begin{scope}
  \definecolor{strokecol}{rgb}{0.0,0.0,0.0};
  \pgfsetstrokecolor{strokecol}
  \draw (27.0bp,92.429bp) node {$A$};
\end{scope}
  % Node: C
\begin{scope}
  \definecolor{strokecol}{rgb}{0.0,0.0,0.0};
  \pgfsetstrokecolor{strokecol}
  \draw (91.458bp,55.215bp) node {$C$};
\end{scope}
  % Node: E
\begin{scope}
  \definecolor{strokecol}{rgb}{0.0,0.0,0.0};
  \pgfsetstrokecolor{strokecol}
  \draw (154.46bp,55.215bp) node {$E$};
\end{scope}
  % Node: D
\begin{scope}
  \definecolor{strokecol}{rgb}{0.0,0.0,0.0};
  \pgfsetstrokecolor{strokecol}
  \draw (27.0bp,18.0bp) node {$D$};
\end{scope}
%
\end{tikzpicture}

 \\
      \hline
      $\mathcal{S}_3: \Theta_A f_2(A, D, E)$ & 
\begin{tikzpicture}[>=latex,line join=bevel,]
  \pgfsetlinewidth{1bp}
%%
\pgfsetcolor{black}
  % Edge: E -- D
  \draw [] (64.122bp,39.432bp) .. controls (60.828bp,37.531bp) and (57.464bp,35.588bp)  .. (54.172bp,33.688bp);
  % Edge: D -- A
  \draw [] (27.0bp,36.017bp) .. controls (27.0bp,47.633bp) and (27.0bp,62.686bp)  .. (27.0bp,74.32bp);
  % Edge: A -- E
  \draw [] (54.336bp,76.647bp) .. controls (57.63bp,74.745bp) and (60.994bp,72.803bp)  .. (64.285bp,70.903bp);
  % Node: A
\begin{scope}
  \definecolor{strokecol}{rgb}{0.0,0.0,0.0};
  \pgfsetstrokecolor{strokecol}
  \draw (27.0bp,92.429bp) node {$A$};
\end{scope}
  % Node: E
\begin{scope}
  \definecolor{strokecol}{rgb}{0.0,0.0,0.0};
  \pgfsetstrokecolor{strokecol}
  \draw (91.458bp,55.215bp) node {$E$};
\end{scope}
  % Node: D
\begin{scope}
  \definecolor{strokecol}{rgb}{0.0,0.0,0.0};
  \pgfsetstrokecolor{strokecol}
  \draw (27.0bp,18.0bp) node {$D$};
\end{scope}
%
\end{tikzpicture}

 \\
      \hline
      & \\
      $\mathcal{S}_4: f_3(D, E)$ & 
\begin{tikzpicture}[>=latex,line join=bevel,]
  \pgfsetlinewidth{1bp}
%%
\pgfsetcolor{black}
  % Edge: D -- E
  \draw [] (54.003bp,18.0bp) .. controls (65.29bp,18.0bp) and (78.387bp,18.0bp)  .. (89.705bp,18.0bp);
  % Node: E
\begin{scope}
  \definecolor{strokecol}{rgb}{0.0,0.0,0.0};
  \pgfsetstrokecolor{strokecol}
  \draw (117.0bp,18.0bp) node {$E$};
\end{scope}
  % Node: D
\begin{scope}
  \definecolor{strokecol}{rgb}{0.0,0.0,0.0};
  \pgfsetstrokecolor{strokecol}
  \draw (27.0bp,18.0bp) node {$D$};
\end{scope}
%
\end{tikzpicture}

 \\
      & \\
      \hline
      & \\
      $\mathcal{S}_5: f_4(E)$ & 
\begin{tikzpicture}[>=latex,line join=bevel,]
  \pgfsetlinewidth{1bp}
%%
\pgfsetcolor{black}
  % Node: E
\begin{scope}
  \definecolor{strokecol}{rgb}{0.0,0.0,0.0};
  \pgfsetstrokecolor{strokecol}
  \draw (27.0bp,18.0bp) node {$E$};
\end{scope}
%
\end{tikzpicture}

 \\
      & \\
    \end{tabular}
    \caption{Grafo de interação resultante da eliminação das variáveis B, C, A, D nessa ordem.}
  \end{center}
\end{table}

\paragraph{
  A Tabela 1 mostra como fica o grafo de interação ao eliminarmos as variáveis B, C, A, D nessa 
  ordem. É importante notar que:
}

\begin{enumerate}
  \item Se $G$ é o grafo de interação dos factors $\mathcal{S}$, então eliminar a variável $\pi(i)$
    de $\mathcal{S}$ leva a construção de um factor sob os vizinhos de $\pi(i)$ em $G$. Por exemplo,
    eliminar a variável $B$ do factor $\mathcal{S}_1$ da Tabela 1 gera um factor sob as variáveis
    $A$, $C$, $D$, que são os vizinhos de $B$ no grafo de interação.
  \item Seja $\mathcal{S}'$ os factors que resultam da eliminação da variável $\pi(i)$ dos factors
    $\mathcal{S}$. Se $G'$ e $G$ são os grafos de interação de $\mathcal{S}'$ e $\mathcal{S}$
    respectivamente, então $G'$ pode ser obtido de $G$ da seguinte forma:
    \begin{enumerate}
      \item Adicione uma aresta em $G$ entre cada par de vizinhos da variável $\pi(i)$ que não 
        estejam já conectados por uma aresta.
      \item Remova a variável $\pi(i)$ de $G$.
    \end{enumerate}
\end{enumerate}

\paragraph{
  Pode-se ver que (a) corresponde a multiplicar os factors que contém a variável $\pi(i)$ em 
  $\mathcal{S}$ e (b) equivale a somar a variável $\pi(i)$ do factor resultante.
}

\paragraph{
  Computar a \textit{width} de uma ordem é útil quando temos um número pequeno de ordens. No
  entanto, quando temos um número muito grande de ordens precisamos achar uma ordem ótima de um
  jeito melhor. Achar uma ordem ótima é NP-difícil (\textit{NP-hard}), mas podemos usar várias
  heurísticas que resultam em boas ordens.
}

\paragraph{
  Uma das heurísticas mais populares é a \textit{min-degree}: sempre elimine a variável que 
  acarreta na construção do menor factor possível. Se usarmos o grafo de interação podemos ver que
  a heurística diz que devemos eliminar a variável que tem o menor número de vizinhos no grafo de
  interação atual. A min-degree é ótima para redes que possuem uma ordem de eliminação de 
  \textit{width} $\leq$ 2.
}

\paragraph{
  Outra heurística que na maioria das vezes é mais efetiva que a min-degree é a \textit{min-fill},
  onde nós eliminamos a variável de tal forma que precisemos adicionar o menor número possível de
  arestas no grafo de interação.
}

\subsection{Computando posterior marginals}

\paragraph{
  Vamos mostrar agora como computar posterior marginals a partir de eliminação de variáveis. Mas
  primeiro vamos definir o que é posterior marginal e joint marginals.
}

\subsubsection{Posterior marginals e joint marginals}

\paragraph{
  Dada uma distribuição de probabilidade conjunta\cite{project-def} $Pr(x_1,...,x_n)$ a 
  distribuição marginal $Pr(x_1,...,x_m), m \leq n$ é definida como:
}

\begin{equation}
  Pr(x_1,...,x_m) = \sum_{x_{m+1},...,x_n} Pr(x_1,...,x_n)
\end{equation}

\paragraph{
  Ou seja, a distribuição marginal pode ser vista como uma \textit{projeção} da distribuição
  conjunta num conjunto menor $X_1,...,X_m$.
}

\paragraph{
  Quando a distribuição marginal é computada dada uma evidência $\textbf{e}$,
}
 
\begin{equation}
  Pr(x_1,...,x_m|\textbf{e}) = \sum_{x_{m+1},...,x_n} Pr(x_1,...,x_n|\textbf{e}),
\end{equation}

\paragraph{
  Então dizemos que ela é uma \textit{posterior marginal}. Quando uma distribuição marginal não é
  dada nenhuma evidência, então dizemos que ela é uma \textit{prior marginal}.
}

\paragraph{
  Vamos agora definir uma marginal conjunta (\textit{joint marginal}). Uma joint marginal é uma
  variação da posterior marginal da forma $P(\textbf{Q}, \textbf{e})$. Ou seja, ao invés de 
  computarmos a probabilidade de \textbf{q} dado \textbf{e}, $Pr(\textbf{q}, \textbf{e})$, nós
  computamos a probabilidade de \textbf{q} e \textbf{e}, $Pr(\textbf{q}, \textbf{e})$. Por exemplo,
  tomemos as Tabelas 2 e 3 dados $\textbf{Q} = \{D, E\}$ e $\textbf{e}: A = true, B = false$ na
  mesma rede da Tabela 1:
}

\begin{table}[h]
  \begin{center}
    \captionsetup{justification=centering}
    \begin{tabular}{*{2}{l} | l}
      D & E & $Pr(\textbf{Q} | \textbf{e})$ \\
      \hline
      true & true & 0.448 \\
      true & false & 0.192 \\
      false & true & 0.112 \\
      false & false & 0.248 \\
    \end{tabular}
    \quad
    \quad
    \begin{tabular}{*{2}{l} | l}
      D & E & $Pr(\textbf{Q}, \textbf{e})$ \\
      \hline
      true & true & 0.21504 \\
      true & false & 0.09216 \\
      false & true & 0.05376 \\
      false & false & 0.11904 \\
    \end{tabular}
    \caption*{Tabelas 2 e 3 Posterior marginal e joint marginal respectivamente.}
  \end{center}
\end{table}
\setcounter{table}{3}

\paragraph{
  A terceira linha da Tabela 3 (joint marginal) diz que:
}

\begin{equation}
  Pr(D = false, E = true, A = true, B = false) = 0.05376
\end{equation}

\paragraph{
  Ao somarmos todas as probabilidades da joint marginal nesse factor, teremos como resultado 0.48,
  que é a probabilidade de evidência $\textbf{e}: A = true, B = false$. Se somarmos todas as 
  probabilidades que aparecem na joint marginal, teremos sempre a probabilidade da evidência 
  \textbf{e}. Isso quer dizer que podemos computar a posterior marginal $Pr(\textbf{Q}|\textbf{e})$
  normalizando a joint marginal $Pr(\textbf{Q}, \textbf{e})$.
}

\subsubsection{Eliminação de variáveis e posterior marginals}

\paragraph{
  Podemos usar eliminação de variáveis para computar joint marginals se zerarmos todas as linhas
  na distribuição de probabilidade conjunta onde há uma inconsistência\cite{report-2} com a 
  evidência \textbf{e}. Mais formalmente dizemos que:
}

\begin{spn-def} A redução de um factor $f(\textbf{X})$ dado uma evidência \textit{e} é um outro
  factor sob as variáveis \textbf{X} denotado por $f^{\textbf{e}}$ e definido como:
  \begin{equation}
    f^{\textbf{e}}(\textbf{x}) \defeq 
    \begin{cases}
      f(\textbf{x}), & \text{se $\textbf{x} \sim \textbf{e}$} \\
      0, & \text{caso contrário.} \\
    \end{cases}
  \end{equation}
\end{spn-def}

\paragraph{
  Portanto, dado o factor da Tabela 2 e evidência $\textbf{e}: E = true$, teríamos:
}

\begin{table}[h]
  \begin{center}
    \begin{tabular}{*{2}{l} | l}
      D & E & $f^\textbf{e}$ \\
      \hline
      true & true & 0.448 \\
      true & false & 0 \\
      false & true & 0.112 \\
      false & false & 0 \\
    \end{tabular}
  \end{center}
\end{table}

\newpage

\paragraph{
  Para facilitar omitimos todas as linhas zeradas:
}

\begin{table}[h]
  \begin{center}
    \begin{tabular}{*{2}{l} | l}
      D & E & $f^\textbf{e}$ \\
      \hline
      true & true & 0.448 \\
      false & true & 0.112 \\
    \end{tabular}
  \end{center}
\end{table}

\paragraph{
  Dado $\textbf{Q} = \{D, E\}$ e $\textbf{e}: A = true, B = false$, a marginal conjunta 
  $Pr(\textbf{Q}, \textbf{e})$ pode ser computada como:
}

\begin{equation}
  Pr(\textbf{Q}, \textbf{e}) = \sum_{A, B, C} (\Theta_{E|C} \Theta_{D|BC} \Theta_{C|A} \Theta_{B|A} \Theta_A)^\textbf{e}
\end{equation}

\paragraph{
  Como pode-se ver, ainda temos o problema da complexidade, já que precisamos multiplicar todas as
  CPTs para depois eliminar as variáveis. No entanto, podemos aplicar o seguinte teorema:
}

\begin{spn-thm} Se $f_1$ e $f_2$ são dois factors e $\textbf{e}$ é uma instância, então:
  \begin{equation}
    (f_1f_2)^\textbf{e} = f_1^\textbf{e}f_2^\textbf{e}
  \end{equation}
\end{spn-thm}

\paragraph{
  Portanto podemos reduzir para:
}

\begin{equation}
  Pr(\textbf{Q} = \{D, E\}, \textbf{e}) = \sum_{A, B, C} \Theta_{E|C}^\textbf{e} \Theta_{D|BC}^\textbf{e}
    \Theta_{C|A}^\textbf{e} \Theta_{B|A}^\textbf{e} \Theta_A^\textbf{e}
\end{equation}

\paragraph{
  Essa forma mantém a distribuição de probabilidade conjunta em forma fatorada e portanto permite
  eliminarmos as variáveis nas CPTs reduzidas.
}

\paragraph{
  Para calcularmos a posterior marginal $Pr(C|A=true)$, dados:
}

\begin{equation}
  \begin{split}
    Pr(C=true, A=true) = 0.192 \\
    Pr(C=false, A=true) = 0.408 \\
    Pr(A=true) = 0.600 
  \end{split}
\end{equation}

\paragraph{
  Precisamos normalizar o factor, ou seja:
}

\begin{table}[h]
  \begin{center}
    \begin{tabular}{l | l}
      C & $Pr(C|A=true)$ \\
      \hline
      true & 0.32 \\
      false & 0.68 \\
    \end{tabular}
  \end{center}
\end{table}

\paragraph{
  Portanto $Pr(C=true|A=true) = 0.32$ e $Pr(C=false|A=true) = 0.68$.
}

\paragraph{
  Note que se tentarmos calcular $Pr(\textbf{Q}|\textbf{e})$ com \textbf{Q} vazio, estaremos
  calculando um factor trivial com valor igual a probabilidade de evidência \textbf{e}.
}

\subsection{Estrutura e complexidade da rede}

\paragraph{
  Apesar de termos duas Redes Bayesianas com o mesmo número de variáveis, as ordens de eliminação
  ótimas de cada uma podem ser diferentes entre si. Isso por causa da \textit{treewidth} das redes.
  Vamos definir o que é uma treewidth, falar sobre NP-completude e NP-dificuldade e em seguida
  classificar as várias estruturas que uma rede pode ter. 
}

\subsubsection{Treewidth}

\paragraph{
  Definimos como treewidth o número que quantifica o quanto uma rede se assemelha à estrutura de 
  uma árvore. Em particular, quanto mais similar a uma árvore, menor o seu treewidth. 
}

\paragraph{
  Um ponto importante sobre treewidth é que nao há nenhuma ordem de eliminação completa onde sua
  \textit{width} é menor que a treewidth da rede. Adicionalmente, existe uma ordem de eliminação
  cuja \textit{width} é igual à treewidth da rede e ela é ótima. No entanto, determinar tal ordem é 
  NP-difícil. Discutiremos mais sobre isso no Apêndice deste relatório.
}

\paragraph{
  Vamos descrever algumas características da treewidth:
}

\begin{itemize}
  \item O número de nós não tem efeito na treewidth.
  \item O número de pais por nó tem efeito direto na treewidth. Se um número de pais por nó chega
    $k$, então o treewidth não é menor que $k$.
  \item Loops tendem a aumentar a treewidth.
  \item O número de loops não tem efeito na treewidth. As interações geradas pelo loop tem efeito
    direto na treewidth.
\end{itemize}

\paragraph{
  Vamos agora falar sobre algumas classes de redes com treewidths conhecidas.
}

\subsubsection{Tree networks, polytrees e multiply connected}

\paragraph{
  \textit{Polytree networks}, também conhecidas como \textit{singly connected networks} são redes
  onde há no máximo um caminho (não direcionado) entre qualquer dois nós. A treewidth dessas redes
  é $k$, onde $k$ é o número máximo de pais que qualquer nó pode ter.
}

\paragraph{
  \textit{Tree networks} são polytrees onde cada nó tem no máximo um pai, e portanto sua treewidth
  é no máximo 1.
}

\paragraph{
  \textit{Multiply connected} são aquelas que não são singly connected.
}

\subsection{Pruning}



\newpage

\printbibliography

\end{document}
