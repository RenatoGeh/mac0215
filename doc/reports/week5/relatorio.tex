\documentclass[a4paper,10pt]{article}

\usepackage[brazilian]{babel}
\usepackage[utf8]{inputenc}
\usepackage[T1]{fontenc}
\usepackage{titlesec}
\usepackage{graphicx}
\usepackage{mathtools}
\usepackage{amsthm}
\usepackage[top=1.0in,bottom=1.0in]{geometry}
\usepackage{hyperref}
\usepackage[singlelinecheck=false]{caption}
\usepackage[backend=biber,url=true,doi=true,eprint=false]{biblatex}
\usepackage{enumitem}
\usepackage[x11names, rgb]{xcolor}
\usepackage{tikz}
\usetikzlibrary{snakes,arrows,shapes}

\addbibresource{../common/references.bib}

\newcommand\blfootnote[1]{%
  \begingroup
  \renewcommand\thefootnote{}\footnote{#1}%
  \addtocounter{footnote}{-1}%
  \endgroup
}

\newcommand\defeq{\mathrel{\overset{\makebox[0pt]{\mbox{\normalfont\tiny\sffamily def}}}{=}}}

\titleformat{\section}
  {\normalfont\scshape\bfseries}{\thesection}{1em}{}
\titleformat{\subsection}
  {\normalfont\scshape\bfseries}{\thesubsection}{1em}{}
\titleformat{\paragraph}
  {\normalfont}{\theparagraph}{1em}{}
\titleformat{\subparagraph}
  {\normalfont}{\thesubparagraph}{1em}{}

\captionsetup[table]{labelsep=space}

\theoremstyle{plain}

\newtheorem*{spn-def}{Definição}
\newtheorem*{spn-thm}{Teorema}

\title{\textbf{Modeling and Reasoning with Bayesian Networks: Inference by Variable Elimination 6.6-6.9}}

\begin{document}
\date{}
\author{}
\vspace*{-40pt}
{\let\newpage\relax\maketitle}

Relatório semana 5 - MAC0215 (Atividade Curricular em Pesquisa)

Aluno: Renato Lui Geh (Bacharelado em Ciência da Computação)

Orientador: Denis Deratani Mauá

\section{Atividades realizadas na semana}

\paragraph{
  Durante a semana foram lidos os seguintes tópicos do livro \textit{Modeling and Reasoning with
  Bayesian Networks}\cite{bayes-net-darwiche}:
}

\begin{description}
  \item[6] - Inference by Variable Elimination
  \begin{description}
    \item[6.6] - Choosing an Elimination Order
    \item[6.7] - Computing Posterior Marginals
    \item[6.8] - Network Structure and Complexity
    \item[6.9] - Query Structure and Complexity
  \end{description}
\end{description}

\section{Definição das atividades}

\paragraph{
  Foram estudados como encontrar a melhor ordem de eliminação de variáveis\cite{report-2}, 
  definimos o que são grafos de interação de Redes Bayesianas, como calcularmos marginais
  posteriores (\textit{posterior marginals}), algumas propriedades da estrutura de uma rede e como 
  podar (\textit{prune}) nós e arestas de uma Rede Bayesiana.
}

\paragraph{
  Esta seção será dividida em subseções onde veremos os itens enumerados no parágrafo acima. A cada
  tópico que não tenha sido apresentado ainda vamos abordar a teoria por trás e introduzir o 
  assunto. Separaremos tais explicações em uma subsubseção equivalente. Abaixo está a lista de 
  tópicos que iremos abordar.
}

\begin{enumerate}
  \item Escolhendo uma ordem de eliminação
    \begin{enumerate}[label*=\arabic*.]
      \item Subgraphs, induced subgraphs, cliques e spanning subgraphs
      \item Grafo de interação
    \end{enumerate}
  \item Computando posterior marginals
    \begin{enumerate}[label*=\arabic*.]
      \item Posterior marginals e joint marginals
    \end{enumerate}
  \item Estrutura e complexidade da rede
    \begin{enumerate}[label*=\arabic*.]
      \item NP-hard e NP-complete
      \item Treewidth
      \item Tree networks, polytrees, trees e multiply connected
    \end{enumerate}
  \item Pruning
    \begin{enumerate}[label*=\arabic*.]
      \item Pruning nodes
      \item Pruning edges
      \item Network pruning
    \end{enumerate}
\end{enumerate}

\subsection{Escolhendo uma ordem de eliminação}

\paragraph{
  Como vimos no relatório anterior\cite{report-2}, queremos criar factors que sejam os menores 
  possíveis. A ordem de eliminação que tiver menor $w$ será a melhor. Dizemos que o $w$ de uma 
  ordem de eliminação é o maior $width$ de todos os factors da ordem de eliminação. 
}

\paragraph{
  Antes de acharmos a melhor ordem de eliminação precisamos saber como calcular o $w$ de uma ordem.
  Um jeito ingênuo de acharmos é, a cada interação $i$ de uma multiplicação e soma, acharmos o 
  $w_i$ com respeito a $\pi(i)$. No final retornamos o maior $w_i$. 
}

\begin{table}[h]
  \begin{center}
    \begin{tabular}{*{4}{l|} l}
      i & $\pi(i)$ & $\mathcal{S}$ & $f_i$ & $w$ \\
      \hline
      & & $\Theta_A \Theta_{B|A} \Theta_{C|A} \Theta_{D|BC} \Theta_{E|C}$ & & \\
      & & & & \\
      1 & B & $\Theta_A \Theta_{C|A} \Theta_{E|C} f_1(A, C, D)$ & $f_1 = \sum_B \Theta_{B|A} \Theta_{D|BC}$ & 3 \\
      2 & C & $\Theta_A f_2(A, D, E)$ & $f_2 = \sum_C \Theta_{C|A} \Theta_{E|C} f_1(A, C, D)$ & 3 \\
      3 & A & $f_3(D, E)$ & $f_3 = \sum_A \Theta_A f_2(A, D, E)$ & 2 \\
      4 & D & $f_4(E)$ & $f_4 = \sum_D f_3(D, E)$ & 1 \\
    \end{tabular}
  \end{center}
\end{table}

\paragraph{
  Na tabela acima a segunda coluna indica a variável $\pi(i)$ que queremos eliminar. A terceira 
  coluna é o resultado da substituição de $\pi(i)$ pelo factor resultante na quarta coluna, onde
  multiplicamos e somamos as CPTs onde $\pi(i)$ está presente. A última coluna é o $w_i$ de cada
  eliminação. Pode-se ver que o $w$ dessa ordem é 3, já que é o maior $w$ entre todas as iterações.
  Note que não precisamos realmente computar a eliminação das variáveis. De fato não fazemos tais
  eliminações, mas a cada $\pi(i)$ que aparecem nos factors $f(\textbf{X}_k)$ substituímos tais
  factors por um novo factor sob as variáveis $\cup_k \textbf{X}_k \setminus {\pi(i)}$.
}

\paragraph{
  Um outro jeito de se achar a $width$ de uma ordem de eliminação é por meio de um grafo não 
  direcionado que representa as interações das CPTs de uma Rede Bayesiana. Antes de definirmos
  esse grafo vamos introduzir alguns conceitos fundamentais sobre grafos.
}

\subsubsection{Subgraphs, induced subgraphs, cliques e spanning subgraphs}

\paragraph{
  Vamos definir $V(G)$ como o conjunto de vértices do grafo G. Similarmente, $E(G)$ é o conjunto de
  arestas do grafo $G$. Dizemos que um grafo é não direcionado se para cada vértice $x, y$ em $G$,
  tanto $xy$ quanto $yx$ são arestas válidas em $G$.
}

\paragraph{
  Primeiro vamos ver a definição de subgrafo:
}

\begin{spn-def} Um grafo $H$ é um subgrafo de $G$ se $V(H)$ é um subconjunto de $V(G)$ e $E(H)$ é
  subconjunto de $E(G)$. Dizemos que $G$ é o supergrafo de $H$ se $H$ é subgrafo de $G$.
\end{spn-def}

\paragraph{
  Vamos agora definir o que é um grafo induzido:
}

\begin{spn-def} Um subgrafo $H$ de um grafo $G$ é dito induzido se para cada par de vértices $x, y$
  em $H$, $xy$ é uma aresta de $H$ se e somente se $xy$ é uma aresta em $G$. Ou seja, $H$ é 
  um subgrafo induzido de $G$ se as arestas em $H$ são exatamente as arestas que aparecem em $G$ 
  sob o mesmo conjunto de vértices. Se $V(H)$ é um subconjunto $S$ de $V(G)$, então $H$ pode ser
  escrito como $G[S]$ e é dito ser induzido por $S$.
\end{spn-def}

\paragraph{
  Agora iremos definir o que é a completude de um grafo:
}

\begin{spn-def} Dizemos que um grafo $G$ é completo se todo par de vértices distintos em $G$ é 
  conectado por uma aresta única.
\end{spn-def}

\paragraph{
  Definimos agora o que é um clique:
}

\begin{spn-def} Um clique é um subconjunto de vértices de um grafo não direcionado tal que o 
  subgrafo induzido dele seja completo. Ou seja, todo par distinto de vértices em um clique é 
  adjacente.
\end{spn-def}

\paragraph{
  Vamos também definir o que é um \textit{spanning subgraph}:
}

\begin{spn-def} Dizemos que um subgrafo $H$ cobre (\textit{spans}) um grafo $G$ se ele tem mesmo
  conjunto de vértices que $G$. $H$ é então dito o \textit{spanning subgraph} de $G$.
\end{spn-def}

\subsubsection{Grafo de interação}

\paragraph{
  Agora que temos uma noção básica sobre grafos podemos definir o grafo de interação.
}

\begin{spn-def} Seja $f_1,...,f_n$ o conjunto de factors. O \textit{grafo de interação} $G$ desses
  factors é um grafo não direcionado construído de tal forma que os nós de $G$ são as variáveis que
  aparecem nos factors $f_1,...,f_n$. Há uma aresta entre duas variáveis em $G$ se e somente se 
  essas variáveis aparecem no mesmo factor.
\end{spn-def}

\paragraph{
  Em outras palavras, as variáveis $\textbf{X}_i$ de cada factor $f_i$ formam um clique em $G$, ou
  seja, as variáveis emparelhadas são adjacentes.
}

\begin{table}[h]
  \begin{center}
    \begin{tabular}{l | c}
      $\mathcal{S}_1: \Theta_A \Theta_{B|A} \Theta_{C|A} \Theta_{D|BC} \Theta_{E|C}$ & \input{graphs/s1-content} \\
      \hline
      $\mathcal{S}_2: \Theta_A \Theta_{C|A} \Theta_{E|C} f_1(A, C, D)$ & \input{graphs/s2-content} \\
      \hline
      $\mathcal{S}_3: \Theta_A f_2(A, D, E)$ & \input{graphs/s3-content} \\
      \hline
      & \\
      $\mathcal{S}_4: f_3(D, E)$ & \input{graphs/s4-content} \\
      & \\
      \hline
      & \\
      $\mathcal{S}_5: f_4(E)$ & \input{graphs/s5-content} \\
      & \\
    \end{tabular}
  \end{center}
\end{table}


\newpage

\printbibliography

\end{document}
