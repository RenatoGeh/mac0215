\documentclass[a4paper,10pt]{article}

\usepackage[brazilian]{babel}
\usepackage[utf8]{inputenc}
\usepackage[T1]{fontenc}
\usepackage{titlesec}
\usepackage{graphicx}
\usepackage{mathtools}
\usepackage{amsthm}
\usepackage[top=1.0in,bottom=1.0in]{geometry}
\usepackage{hyperref}
\usepackage[singlelinecheck=false]{caption}
\usepackage[backend=biber,url=true,doi=true,eprint=false]{biblatex}
\usepackage{enumitem}
\usepackage[x11names, rgb]{xcolor}
\usepackage{tikz}
\usetikzlibrary{snakes,arrows,shapes}

\addbibresource{../common/references.bib}

\newcommand\blfootnote[1]{%
  \begingroup
  \renewcommand\thefootnote{}\footnote{#1}%
  \addtocounter{footnote}{-1}%
  \endgroup
}

\newcommand\defeq{\mathrel{\overset{\makebox[0pt]{\mbox{\normalfont\tiny\sffamily def}}}{=}}}

\titleformat{\section}
  {\normalfont\scshape\bfseries}{\thesection}{1em}{}
\titleformat{\subsection}
  {\normalfont\scshape\bfseries}{\thesubsection}{1em}{}
\titleformat{\paragraph}
  {\normalfont}{\theparagraph}{1em}{}
\titleformat{\subparagraph}
  {\normalfont}{\thesubparagraph}{1em}{}

\captionsetup[table]{labelsep=space}

\theoremstyle{plain}

\newtheorem*{spn-def}{Definição}
\newtheorem*{spn-thm}{Teorema}

\title{\textbf{Uma Introdução a Sum-Product Networks}}

\begin{document}
\date{}
\author{}
\vspace*{-40pt}
{\let\newpage\relax\maketitle}

Relatório semana 9 - MAC0215 (Atividade Curricular em Pesquisa)

Aluno: Renato Lui Geh (Bacharelado em Ciência da Computação)

Orientador: Denis Deratani Mauá

\section{Atividades realizadas na semana}

\paragraph{
  Durante a semana foram lidos as seguintes partes dos papers abaixo:
}

\begin{itemize}
  \item \textit{Learning the Structure of Sum-Product Networks}, [R. Gens, P. Domingos]
    \cite{gens-domingos}
    \begin{itemize}
      \item Introduction
      \item Sum-Product Networks
    \end{itemize}
  \item \textit{Sum-Product Networks: A New Deep Architecture}, [H. Poon, P. Domingos]
    \cite{poon-domingos}
    \begin{itemize}
      \item Introduction
      \item Sum-Product Networks
      \item Sum-Product Networks and other models
    \end{itemize}
\end{itemize}

\section{Definição das atividades}

\paragraph{
  Os tópicos mencionados na seção anterior referem-se à definição de uma Sum-Product Network e
  citam algumas semelhanças com outros modelos probabilísticos assim como suas diferenças.
}

\paragraph{
  Neste relatório vamos definir o que são Sum-Product Networks de uma forma mais didática e vamos
  supor que o leitor tenha conhecimento prévio de todo conteúdo coberto nos relatórios anteriores.
  Após termos definido Sum-Product Networks, vamos ver algumas propriedades e teoremas relacionados
  e em seguida vamos comparar, de forma sucinta, com outros modelos probabilísticos.
}

\paragraph{
  Vamos separar esta seção nos seguintes tópicos:
}

\begin{enumerate}
  \item Introdução
  \item Definição
  \item Propriedades
  \item Comparação
\end{enumerate}

\subsection{Introdução}



\subsection{Definição}

\newpage

\printbibliography

\end{document}
