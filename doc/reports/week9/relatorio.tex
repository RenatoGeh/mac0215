\documentclass[a4paper,10pt]{article}

\usepackage[brazilian]{babel}
\usepackage[utf8]{inputenc}
\usepackage[T1]{fontenc}
\usepackage{titlesec}
\usepackage{graphicx}
\usepackage{mathtools}
\usepackage{amsthm}
\usepackage[top=1.0in,bottom=1.0in]{geometry}
\usepackage{hyperref}
\usepackage[singlelinecheck=false]{caption}
\usepackage[backend=biber,url=true,doi=true,eprint=false]{biblatex}
\usepackage{enumitem}
\usepackage[x11names, rgb]{xcolor}
\usepackage{tikz}
\usetikzlibrary{snakes,arrows,shapes}

\addbibresource{../common/references.bib}

\newcommand\blfootnote[1]{%
  \begingroup
  \renewcommand\thefootnote{}\footnote{#1}%
  \addtocounter{footnote}{-1}%
  \endgroup
}

\newcommand\defeq{\mathrel{\overset{\makebox[0pt]{\mbox{\normalfont\tiny\sffamily def}}}{=}}}

\titleformat{\section}
  {\normalfont\scshape\bfseries}{\thesection}{1em}{}
\titleformat{\subsection}
  {\normalfont\scshape\bfseries}{\thesubsection}{1em}{}
\titleformat{\paragraph}
  {\normalfont}{\theparagraph}{1em}{}
\titleformat{\subparagraph}
  {\normalfont}{\thesubparagraph}{1em}{}

\captionsetup[table]{labelsep=space}

\theoremstyle{plain}

\newtheorem*{spn-def}{Definição}
\newtheorem*{spn-thm}{Teorema}

\title{\textbf{Uma Introdução a Sum-Product Networks}}

\begin{document}
\date{}
\author{}
\vspace*{-40pt}
{\let\newpage\relax\maketitle}

Relatório semana 9 - MAC0215 (Atividade Curricular em Pesquisa)

Aluno: Renato Lui Geh (Bacharelado em Ciência da Computação)

Orientador: Denis Deratani Mauá

\section{Atividades realizadas na semana}

\paragraph{
  Durante a semana foram lidos as seguintes partes dos papers abaixo:
}

\begin{itemize}
  \item \textit{Learning the Structure of Sum-Product Networks}, [R. Gens, P. Domingos]
    \cite{gens-domingos}
    \begin{itemize}
      \item Introduction
      \item Sum-Product Networks
    \end{itemize}
  \item \textit{Sum-Product Networks: A New Deep Architecture}, [H. Poon, P. Domingos]
    \cite{poon-domingos}
    \begin{itemize}
      \item Introduction
      \item Sum-Product Networks
      \item Sum-Product Networks and other models
    \end{itemize}
\end{itemize}

\section{Definição das atividades}

\paragraph{
  Os tópicos mencionados na seção anterior referem-se à definição de uma Sum-Product Network e
  citam algumas semelhanças com outros modelos probabilísticos assim como suas diferenças.
}

\paragraph{
  Neste relatório vamos definir o que são Sum-Product Networks de uma forma mais didática e vamos
  supor que o leitor tenha conhecimento prévio de todo conteúdo coberto nos relatórios anteriores.
  Após termos definido Sum-Product Networks, vamos ver algumas propriedades e teoremas relacionados
  e em seguida vamos comparar, de forma sucinta, com outros modelos probabilísticos.
}

\paragraph{
  Vamos separar esta seção nos seguintes tópicos:
}

\begin{enumerate}
  \item Introdução
    \begin{enumerate}[label*=\arabic*.]
      \item Distribuição normalizada de produtos de factors
      \item Função de partição
    \end{enumerate}
  \item Definição
  \item Propriedades
  \item Comparação
\end{enumerate}

\subsection{Introdução}

\paragraph{
  Um dos maiores problemas com modelos gráficos é a intractabilidade da inferência e aprendizado
  da estrutura. Inferência é sempre exponencial no pior caso, e como aprendizado usa inferência,
  a complexidade continua intratável. Além do mais, a amostragem necessária para aprendizado
  preciso é também exponencial no pior dos casos no tamanho do escopo. De fato existem modelos
  gráficos onde a inferência é tratável, no entanto elas são limitadas quanto às representações
  de distribuições de forma compacta.
}

\paragraph{
  Vamos mostrar que Sum-Product Networks (SPN), um novo tipo de arquitetura profunda, permite que
  computemos a função partição, a probabilidade de evidência e o estado MAP\cite{report-1} com
  complexidade linear no número de arestas da SPN. Também vamos definir validade de uma SPN assim
  como completude e consistência. Depois vamos mostrar outras definições assim como alguns teoremas
  derivados dessas propriedades.
}

\paragraph{
  Antes de começarmos a definir Sum-Product Networks, precisamos antes explicar o que é uma
  distribuição normalizada de produtos de factors e definir uma função de partição.
}

\subsubsection{Distribuição normalizada de produtos de factors}

\paragraph{
  O objetivo de modelos gráficos probabilísticos é representar distribuições de forma compacta.
  Podemos representar tais distribuições como um produto normalizado dos factors\cite{report-2}
  envolvidos. Tal representação é um jeito compacto de se representar as CPTs envolvidas.
}

\begin{spn-def} Sejam $x \in \mathcal{X}$ um vetor $d$-dimensional representando uma instância de
  $d$ variáveis, $\phi_k$ uma função potential\cite{report-2} do subconjunto $x_{\{k\}}$ de
  variáveis (ou seja, seu escopo\cite{project-def}) e $Z$ a função partição que veremos mais a
  frente. Representamos distribuições compactamente como o seguinte produto normalizado:
  \begin{equation}
    P(X = x) = \frac{1}{Z} \prod_k \phi_k (x_{\{k\}})
  \end{equation}
\end{spn-def}

\paragraph{
  A representação acima é dita normalizada pois queremos representa-la como uma probabilidade, ou
  seja, um número real no intervalo $[0, 1]$. Como pode-se notar, dividimos o produtório por $Z$,
  a chamada função partição. De fato, como veremos a seguir, a função partição normaliza o produto
  dos factors.
}

\subsubsection{Função de partição}

\paragraph{
  Dizemos a função partição uma função que toma como argumentos todos os estados das variáveis e
  retorna a soma de todos os produtórios de todos os factors de cada estado.
}

\begin{spn-def} Seja $\phi_k$ uma função potential, dizemos que a função partição é
  \begin{equation}
    Z = \sum_{x \in X} \prod_k \phi_k (x_{\{k\}})
  \end{equation}
\end{spn-def}

\paragraph{
  Portanto, é fácil notar que $\frac{1}{Z} \prod_k \phi_k (x_{\{k\}})$ é uma normalização por $Z$,
  já que $Z$ é a soma de todos os possíveis resultados do produtório, e portanto será sempre maior
  ou igual ao valor do produtório normalizado, levando a $0 \leq P(X = x) \leq 1$, assumindo-se que
  $\phi_i \geq 0$.
}

\paragraph{
  No caso de $Z=1$, então temos o caso trivial onde o produtório dos factors dada uma instância já
  está dentro do intervalo $[0,1]$.
}

\paragraph{
  Uma das dificuldades de se computar inferência em modelos gráficos é a intractabilidade de $Z$,
  já que $Z$ é a soma de um número exponencial de termos. Como todas as marginals\cite{report-5}
  são somas de subconjuntos desses termos, computa-las é igualmente intratável. No entanto, se
  acharmos uma maneira eficiente de computar $Z$, então também podemos computar as marginals
  eficientemente. Mas $Z$ é computado apenas com somas e produtos, e pode ser eficientemente
  computado se aplicarmos a distributiva em $Z$ de tal forma que envolvamos um número polinomial de
  somas e produtos.
}

\subsection{Definição}

\paragraph{
  Assim como em [P. Domingos, H. Poon]\cite{poon-domingos}, vamos introduzir Sum-Product Networks
  com variáveis Booleanas. Mais para frente veremos que para variáveis discretas ou contínuas o
  processo é similar.
}

\paragraph{
  Antes de definirmos SPNs, vamos introduzir algumas notações:
}

\begin{itemize}
  \item A negação de $X_i$ é representada por $\overline{X_i}$.
  \item A função indicadora\cite{report-1} $[.]$ tem valor 1 se seu argumento é $true$ e 0 caso
    contrário.
  \item Abreviaremos $[X_i]$ por $x_i$ e $[\overline{X}_i]$ por $\overline{x_i}$.
\end{itemize}



\newpage

\printbibliography

\end{document}
