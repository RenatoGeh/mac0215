\documentclass[a4paper,10pt]{article}

\usepackage[brazilian]{babel}
\usepackage[utf8]{inputenc}
\usepackage[T1]{fontenc}
\usepackage{titlesec}
\usepackage{graphicx}
\usepackage{mathtools}
\usepackage{amsthm}
\usepackage[top=1.0in,bottom=1.0in]{geometry}
\usepackage{hyperref}
\usepackage[singlelinecheck=false]{caption}
\usepackage[backend=biber,url=true,doi=true,eprint=false]{biblatex}
\usepackage{enumitem}
\usepackage[x11names, rgb]{xcolor}
\usepackage{tikz}
\usetikzlibrary{snakes,arrows,shapes}

\addbibresource{../common/references.bib}

\newcommand\blfootnote[1]{%
  \begingroup
  \renewcommand\thefootnote{}\footnote{#1}%
  \addtocounter{footnote}{-1}%
  \endgroup
}

\newcommand\defeq{\mathrel{\overset{\makebox[0pt]{\mbox{\normalfont\tiny\sffamily def}}}{=}}}

\titleformat{\section}
  {\normalfont\scshape\bfseries}{\thesection}{1em}{}
\titleformat{\subsection}
  {\normalfont\scshape\bfseries}{\thesubsection}{1em}{}
\titleformat{\paragraph}
  {\normalfont}{\theparagraph}{1em}{}
\titleformat{\subparagraph}
  {\normalfont}{\thesubparagraph}{1em}{}

\captionsetup[table]{labelsep=space}

\theoremstyle{plain}

\newtheorem*{spn-def}{Definição}
\newtheorem*{spn-thm}{Teorema}

\title{\textbf{Aprendizado Automático de Sum-Product Networks}}

\begin{document}
\date{}
\author{}
\vspace*{-40pt}
{\let\newpage\relax\maketitle}

Relatório final - MAC0215 (Atividade Curricular em Pesquisa)

Aluno: Renato Lui Geh (Bacharelado em Ciência da Computação)

Orientador: Denis Deratani Mauá

\section{Introdução}

\paragraph{
  O intuito deste projeto de Iniciação Científica, como visto no relatório inicial do projeto
  \cite{project-def}, era aprender a estrutura e propriedades de uma Sum-Product Network, um Modelo
  Gráfico Probabilístico; entender a motivação por trás do estudo de tais redes probabilísticas,
  como Redes Bayesianas; adquirir toda a base matemática por trás destas redes e finalmente
  conseguir entender como aplicar aprendizado e inferência em Sum-Product Networks.
}

\paragraph{
  Com o tempo disposto neste semestre, conseguiu-se estudar grande parte da área teórica por trás
  de Sum-Product Networks, lendo-se a base matemática por trás de Redes Bayesianas e Sum-Product
  Networks -- como será mostrado nas seções seguintes -- e entendendo conceitos de Teoria de Grafos
  e probabilidade lendo-se tanto Darwiche\cite{bayes-net-darwiche} quanto estudando-se a matéria
  dada em MAC0425 Inteligência Artificial; estudou-se definições, teoremas e propriedades de
  Sum-Product Networks em dois diferentes artigos\cite{poon-domingos}\cite{gens-domingos}; também
  foi estudado métodos de aprendizagem, como Regressão Linear, Naive Bayes, Nearest Neighbour e
  Decision Trees tanto por estudo próprio quanto pelas aulas dadas em MAC0425; e finalmente foi
  estudado um método de aprendizado em Sum-Product Networks em \cite{poon-domingos}.
}

\paragraph{
  Como a maioria dos estudos feitos para este projeto são de caráter teórico, as evidências de
  estudo mais concretas são os relatórios publicados semanalmente. Na seção seguinte detalhar-se-á
  os estudos citados nesta seção.
}

\section{Histórico}

\paragraph{
  Para cada subseção nesta seção existe um relatório associado explicitando o trabalho em maiores
  detalhes. Cada subseção equivale a uma semana. Cada semana possue um relatório. O intuito de se
  fazer um relatório para cada semana é manter um resumo de toda a matéria vista na semana, além
  de servir como ``notas de aula'' sobre o que foi lido. Como o que é transcrito para os relatórios
  é um resumo da matéria, este também serve como um jeito de servir como referência posteriormente.
  Adicionalmente, tais relatórios também servem como uma forma gradual e didática de se aprender
  Sum-Product Networks, simulando os meus ``passos''.
}

\paragraph{
  Cada subseção está organizada em diferentes blocos enumerados a seguir:
}

\begin{itemize}
  \item Breve resumo da semana/relatório.
  \item Enumeração dos assuntos vistos.
  \item Resumo dos assuntos.
\end{itemize}

\paragraph{
  A breve descrição da semana/relatório consiste apenas na explicação do que foi feito na semana e
  eventualmente resumido no relatório. A discussão detalhada dos assuntos entrará em maiores
  detalhes sobre o estudo, explicando o que foi feito de concreto na semana. O resumo do relatório
  é mais técnico e entra em mais detalhes na matéria.
}

\subsection{Aprendizado Automático de Sum-Product Networks (SPN)}

\paragraph{
  Nesta semana foi feita a definição do projeto (relatório inicial).\cite{project-def}
}

\paragraph{
  Durante essa semana pesquisou-se o que Sum-Product Networks de fato são, foi estudada a base
  mínima de probabilidade necessária e foram lidas as aplicações de Sum-Product Networks. Os
  seguintes tópicos foram estudados:
}

\begin{itemize}
  \item Distribuições de probabilidade multivariadas.
  \item Escopo de uma distribuição.
  \item Espaço de uma distribuição multivariada.
  \item Inferência.
  \item Definição de uma Sum-Product Network.
  \item Vantagens de Sum-Product Networks.
  \item Aplicações de Sum-Product Networks.
\end{itemize}

\paragraph{
  O relatório associado a essa semana que contém os tópicos citados acima é \textit{Aprendizado
  Automático de Sum-Product Networks (SPN)}\cite{project-def}.
}

\paragraph{
  Esta semana foi importante para ter a base matemática para entender Sum-Product Networks. SPNs
  são redes probabilísticas que podem ser representadas como um digrafo acíclico enraizado com
  folhas representando probabilidades. Como cada folha é uma distribuição monovariável, então uma
  SPN que tenha mais de uma folha e tenha ao menos um nó produto é uma distribuição de
  probabilidade multivariada. Isso ocorre por causa da definição de consistência de uma SPN, que
  pode ser lida em \cite{report-9}. Além disso este relatório representa os primeiros contatos com
  SPNs, estudando-se definição (superficialmente), vantagens de SPNs sobre outras redes e
  aplicações mais práticas.
}

\subsection{Modeling and Reasoning with Bayesian Networks: Compiling Bayesian Networks}

\paragraph{
  Esta semana foi a primeira das muitas em que foi lido o livro \textit{Modeling and Reasoning with
  Bayesian Networks}\cite{bayes-net-darwiche}. Nesta semana particularmente foi lido o capítulo
  12 ``Compiling Bayesian Networks''.
}

\paragraph{
  Podemos enumerar os tópicos estudados nesta semana da seguinte forma:
}

\begin{itemize}
  \item Redes Bayesianas.
  \item CPT, MPE e MAP.
  \item Consistência de variáveis dada uma evidência.
  \item Network polynomial.
  \item Circuit propagation e derivadas parciais.
\end{itemize}

\paragraph{
  O relatório associado a essa semana que contém os tópicos citados acima é \textit{Modeling and
  Reasoning with Bayesian Networks: Compiling Bayesian Networks}\cite{report-1}.
}

\paragraph{
  O capítulo ``Compiling Bayesian Networks'' pode ser visto como um precursor para a criação de
  Sum-Product Networks. Neste capítulo Darwiche propõe um método de se representar uma Rede
  Bayesiana como um digrafo acíclico enraizado onde as folhas são parâmetros ou indicadores e os
  nós internos são somas ou produtos de tal forma que o resultado deste circuito aritmético na raíz
  é a network polynomial da distribuição.
}

\paragraph{
  Poon e Domingos expandem esta ideia em seu artigo \textit{Sum-Product Networks: A New Deep
  Architecture}\cite{poon-domingos} com mais formalização e com a diferença das folhas, ao invés
  de serem ou parâmetros ou indicadores, serem distribuições monovariáveis com os pesos nas arestas
  originárias de nós somas.
}

\paragraph{
  Também foi estudado o que significa uma variável da rede estar consistente com um conjunto de
  indicadores. Esse conjunto de indicadores pode ser chamado de evidência, e representa o
  conhecimento \textit{a priori} de mundo. O conceito de consistência de uma variável dada uma
  evidência é muito importante para SPNs.
}

\paragraph{
  Neste capítulo temos o primeiro contato com network polynomials, um conceito fundamental para o
  entendimento de Sum-Product Networks. Além disso, foram estudados algumas nomenclaturas usadas
  em probabilidade, como CPT, MPE e MAP (seus significados podem ser vistos no relatório). Além
  disso foi estudado Redes Bayesianas superficialmente para podermos entender como compila-las em
  circuitos aritméticos.
}

\paragraph{
  Outro ponto fundamental foi o estudo de propagação do circuito. Propagação do circuito significa
  usar as derivadas parciais das variáveis de tal forma que não precisemos recalcular os parâmetros
  do circuito aritmético se mudarmos a evidência. No final do relatório também analisamos a
  complexidade dos algoritmos de \textit{bottom-up} (calcular inferência) e \textit{top-down}
  (calcular as derivadas parciais).
}

\subsection{Modeling and Reasoning with Bayesian Networks: Inference by Variable Elimination 6.1-6.5}

\paragraph{
  Nesta semana foi lida a primeira parte do capítulo 6 do livro \textit{Modeling and Reasoning with
  Bayesian Networks}\cite{bayes-net-darwiche}. A primeira parte consiste nas subseções 1-5.
}

\paragraph{
  Os tópicos 6.1-6.5 dizem respeito aos seguintes assuntos:
}

\begin{itemize}
  \item Factors.
  \item Soma e multiplicação de factors.
  \item Eliminação de variáveis.
  \item Ordem de eliminação.
\end{itemize}

\paragraph{
  O relatório associado a essa semana que contém os tópicos citados acima é \textit{Modeling and
  Reasoning with Bayesian Networks: Inference by Variable Elimination 6.1-6.5}\cite{report-2}.
}

\paragraph{
  Eliminação de variáveis é um método por estrutura de simplificar uma rede para torna-la mais
  tratável. Ao contrário da estrutura local (que será visto mais a frente), métodos por estrutura
  não usam os parâmetros dados pela query na hora da inferência. Isso as torna mais gerais, mas
  não possuem melhor performance quando comparadas a métodos que usam estrutura local.
}

\paragraph{
  Factors podem ser vistos como funções que mapeiam cada instância das variável em um número
  não-negativo. Também chamados de potenciais, factors servem para representar um jeito mais
  compacto de se fazer operações em CPTs.
}

\paragraph{
  Soma (\textit{summing out}) de uma variável $X$ é uma operação para se tentar eliminar a
  referência de uma variável em uma CPT. Multiplicação pode ser visto como a operação de união de
  variáveis.
}

\paragraph{
  Eliminação de variáveis consiste no uso de multiplicação e soma de variáveis para se descartar
  uma variável da Rede Bayesiana. Foram vistos algumas propriedades, algoritmos e a definição de
  se eliminar variáveis.
}

\paragraph{
  A ordem da eliminação de cada variável influencia na complexidade do algoritmo. Quer-se sempre
  eliminar variáveis em uma ordem menos custosa. Foi visto um algoritmo bem simples mas intuitivo
  de como achar a menor ordem de eliminação, mas este algoritmo era custoso, já que precisava
  calcular cada ordem individualmente.
}

\subsection{Modeling and Reasoning with Bayesian Networks: Inference by Variable Elimination 6.6-6.9}

\paragraph{
  Nesta semana foi lida a segunda parte do capítulo 6 do livro \textit{Modeling and Reasoning with
  Bayesian Networks}\cite{bayes-net-darwiche}. A segunda parte consiste nas subseções 6-9.
}

\paragraph{
  Os tópicos 6.6-6.9 dizem respeito aos seguintes assuntos:
}

\begin{itemize}
  \item Escolhendo uma ordem de eliminação.
  \item Computando marginais posteriores.
  \item Estrutura e complexidade da rede.
  \item Estrutura e complexidade da query.
\end{itemize}

\paragraph{
  O relatório associado a essa semana que contém os tópicos citados acima é \textit{Modeling and
  Reasoning with Bayesian Networks: Inference by Variable Elimination 6.6-6.9}\cite{report-5}.
}

\paragraph{
  Além do que foi lido no livro, também foram pesquisados os seguintes assuntos:
}

\begin{itemize}
  \item Subgrafos, subgrafos induzidos, cliques, spanning subgrafos.
  \item Tree networks, polytrees e multiply connected.
  \item P, NP, NP-dificuldade e NP-completude.
\end{itemize}

\paragraph{
  Conforme os tópicos 6.6-6.9 foram lidos, assuntos que fossem dependências do que o autor citava
  foram pesquisados e adicionados no relatório. Portanto, os itens da segunda lista dessa seção
  estão presentes no relatório, com explicações e definições.
}

\paragraph{
  Primeiro foram vistos dois algoritmos não tão triviais para se achar uma melhor ordem de
  eliminação. Os dois algoritmos são similares, suas únicas diferenças sendo a heurística usada.
  A primeira tenta construir uma ordem que tenha menor fator possível. A segunda tenta adicionar
  o menor número possível de arestas no grafo de interação.
}

\paragraph{
  Para entender o grafo de interação, foi preciso primeiro pesquisar sobre subgrafos, subgrafos
  induzidos, cliques e spanning subgrafos. Um grafo de interação tem seus nós como variáveis que
  aparecem nos factors. Existe uma aresta entre dois nós se as duas variáveis aparecem no mesmo
  factor. Pode-se ver este grafo como um clique, tornando o problema de se achar uma ordem um
  problema de busca e eliminação de cliques. No entanto, achar uma boa ordem é NP-difícil.
}

\paragraph{
  Para se entender o que significa NP-difícil, foi pesquisado P, NP, NP-dificuldade e
  NP-completude. Este estudo foi adicionado no apêndice do relatório por ser mais geral que no caso
  do estudo sobre subgrafos.
}

\paragraph{
  O próximo tópico do livro era como se computar marginais posteriores. Foi então pesquisado o que
  significavam marginais posteriores e marginais conjuntas. Para isso foi usado o próprio livro do
  Darwiche\cite{bayes-net-darwiche} que definia-as. Em seguida foi visto como usar eliminação de
  variáveis para computar marginais conjuntas. Também foram vistas várias notações.
}

\paragraph{
  O seguinte assunto era sobre estrutura e complexidade da rede. A complexidade de duas Redes
  Bayesianas não se restringe apenas ao número de variáveis, mas também a sua estrutura. Um mesmo
  número de variáveis pode ter ordens ótimas de eliminação com diferentes complexidades. Isso se
  deve ao fator de treewidth da rede.
}

\paragraph{
  Para se entender treewidth foi preciso estudar tree networks, polytrees e multiply connected. O
  livro menciona de forma superficial o que elas são, sendo suas definições postas no relatório.
}

\paragraph{
  Em seguida foi visto um jeito de se tentar melhorar a estrutura de uma rede para simplificar a
  complexidade. Foi lido sobre como se efetuar um corte em uma árvore, o que inclue aparar tanto
  nós quanto arestas da Rede Bayesiana.
}

\subsection{Modeling and Reasoning with Bayesian Networks: Inference with Local Structure 13.1-13.3}

\paragraph{
  Nesta semana foram lidos os tópicos 1-3 do capítulo 13 do livro \textit{Modeling and Reasoning
  with Bayesian Networks}\cite{bayes-net-darwiche}.
}

\paragraph{
  Os seguintes assuntos foram estudados:
}

\begin{itemize}
  \item Comparação entre estrutura local e baseado em estrutura.
  \item Impacto de estrutura local na complexidade da inferência.
  \item Independência contexto-específica.
  \item Determinismo.
  \item Evidência em estrutura local.
\end{itemize}

\paragraph{
  O relatório associado a essa semana que contém os tópicos citados acima é \textit{Modeling and
  Reasoning with Bayesian Networks: Inference with Local Structure 13.1-13.3}\cite{report-8}.
}

\paragraph{
  Estrutura local é o uso dos parâmetros da rede para melhorarmos a complexidade da inferência.
  Em baseado em estrutura, inferência de uma rede é limitado plea treewidth da rede. No entanto,
  com estrutura local, podemos computar em tempo tratável uma rede com treewidth intratável.
}

\paragraph{
  Foram lidos alguns exemplos de uso de estrutura local para deixar a rede mais simples. Uma delas
  é com o uso de restrições lógicas. Se um nó pai tem uma relação de ou-lógico com seus filhos,
  então podemos simplificar a rede fatorando a network polynomial de tal forma que tenhamos um
  circuit aritmético tratável, mesmo que o número de filhos seja exponencial.
}

\paragraph{
  Em seguida foi lido sobre independência contexto-específica. A relação entre nós de tal forma que
  podemos calcular a probabilidade de uma variável independentemente do valor dos outros nós se
  chama independência contexto-específica.
}

\paragraph{
  O próximo assunto tratava sobre determinismo de uma rede, que é uma relação de estrutura local
  onde os parâmetros da rede tem valor zero.
}

\paragraph{
  Finalmente, foi visto sobre como usar evidência como um jeito de se simplificar a rede. Usar a
  evidência de uma rede é um jeito de se fazer estrutura local.
}

\subsection{Uma Introdução a Sum-Product Networks}

\paragraph{
  Nesta semana dediquei-me a escrita de um relatório onde, baseado em dois papers (Domingos, Poon
  \cite{poon-domingos} e Domingos, Gens \cite{gens-domingos}), pretende introduzir Sum-Product
  Networks de forma que, lendo-se todos os relatórios anteriores, consiga entender o que são
  Sum-Product Networks. A ênfase desta semana foi na definição e nas propriedades mais importantes
  de uma Sum-Product Network. Posteriormente planeja-se completar o relatório com os outros
  assuntos mencionados nesses e em outros papers.
}

\paragraph{
  Os tópicos incluídos neste relatório foram:
}

\begin{itemize}
  \item Distribuição normalizada de produtos de factors.
  \item Função de partição.
  \item Definições de Sum-Product Networks.
  \item Validade, completude e consistência de uma Sum-Product Network.
\end{itemize}

\paragraph{
  O relatório associado a essa semana que contém os tópicos citados acima é \textit{Uma Introdução
  a Sum-Product Networks}\cite{report-9}.
}

\paragraph{
  Foram lidos os papers \textit{Sum-Product Networks: A New Deep Architecture}\cite{poon-domingos}
  de P. Domingos e H. Poon e \textit{Learning the Structure of Sum-Product Networks}
  \cite{gens-domingos} de R. Gens e P. Domingos. Em particular, foi lida a introdução, definição
  e em seguida propriedades de Sum-Product Networks dos dois papers. Em Domingos e Poon a definição
  de SPNs é mais geral, enquanto que Gens e Domingos assumem uma SPN que seja sempre válida. No
  relatório foi escrita primeiro a definição em Domingos e Poon e, após termos uma noção do que
  são SPNs, a de Gens e Domingos.
}

\paragraph{
  Antes de entrarmos em detalhes sobre SPNs, precisou-se estudar o que são distribuições
  normalizadas de produtos de factors e funções partição. Foi lida a introdução contida em Poon e
  Domingos como referência. Após estudarmos a base, foram escritas as duas definições de
  Sum-Product Networks.
}

\paragraph{
  Em seguida foi lido sobre as três principais propriedades em SPNs: validade, completude e
  consistência. Uma SPN válida é importante pois computa a probabilidade de uma evidência em tempo
  linear e de forma correta. Uma SPN inválida sempre tem valor aproximado.
}

\paragraph{
  Finalmente, leu-se sobre o primeiro teorema em Sum-Product Networks, que dita que uma SPN é
  válida se ela é completa e consistente. Em seguida foi feita a prova deste teorema usando o
  paper de Poon e Domingos como ponto de partida.
}

\subsection{Aprendizado Automático de Sum-Product Networks (Pôster)}

\paragraph{
  Nesta semana foi feito o pôster para MAC0215.
}

\begin{itemize}
  \item Motivações.
  \item Definições.
  \item Propriedades.
  \item Aprendizado.
  \item Algoritmo de Aprendizado.
  \item Experimentos.
\end{itemize}

\paragraph{
  O pôster associado aos tópicos acima é \textit{Aprendizado Automático de Sum-Product Networks}
  \cite{poster-spn}.
}

\paragraph{
  Para fazer o pôster releu-se os dois artigos \cite{poon-domingos} e \cite{gens-domingos}. Também
  foi feita uma breve pesquisa na motivação dos criadores de se criar Sum-Product Networks. Em
  seguida foi relida as definições dadas pelos dois artigos e transcrita de maneira mais sucinta
  possível para o pôster. Tentou-se manter as propriedades claras mas ao mesmo tempo pequenas e
  fáceis de ler e em seguida foi dada uma explicação da razão pela qual deseja-se validade em uma
  SPN.
}

\paragraph{
  Até este ponto do projeto não havia sido lido nada sobre aprendizado em Sum-Product Networks.
  Antes de entrar neste ponto era preciso saber toda a base que foi visto nos relatórios
  anteriores. Para fazer o pôster, no entanto, era preciso ter a ideia de aprendizado de SPNs em
  mente, e portanto foi lida a parte sobre aprendizado em Poon e Domingos \cite{poon-domingos} e
  também em Gens e Domingos \cite{gens-domingos}.
}

\paragraph{
  Foi feita uma divisão entre dois tipos de aprendizado em SPNs. A primeira refere-se ao
  aprendizado dos pesos de uma SPN já existente. Já na segunda aprende-se tanto a estrutura quanto
  os parâmetros da SPN. O primeiro algoritmo de aprendizado em SPNs pertence a primeira classe e
  foi dada em \cite{poon-domingos}. Gens e Domingos criaram o primeiro algoritmo da segunda classe
  em \cite{gens-domingos}.
}

\paragraph{
  Decidiu-se estudar mais profundamente o algoritmo dado por Poon e Domingos por ser mais simples
  e também por ter experimentos mais concretos. Foi lido sobre Gradient Descent e
  Expectation-Maximization, duas técnicas de otimização usadas para atualizar e inferir os pesos
  da SPN durante o aprendizado. Em seguida foi lido sobre como o algoritmo funcionava.
}

\paragraph{
  Para ter resultados mais concretos para o pôster, foram simulados os experimentos citados em
  \cite{poon-domingos}. As imagens sobre compleção de imagem e a tabela de taxa de acerto em
  reconhecimento de imagens no poster fazem parte do conjunto de experimentos feitos no artigo.
}

\paragraph{
  Finalmente, foi discutido com o orientador sobre quais seriam os futuros trabalhos que poderiam
  ser feitos no projeto, adicionando isso ao pôster.
}

\subsection{Palestras e relatório final}

\paragraph{
  Nesta semana foram assistidas palestras na matéria de MAC0425 Inteligência Artificial e foi
  escrito o relatório final.
}

\begin{itemize}
  \item \textit{Aprendizagem de redes bayesianas} - Walter Perez
  \item \textit{Redes neurais e Deep Learning} - Jessica de Souza
  \item \textit{Sum-Product Networks} - Julissa Villanueva
  \item \textit{Algoritmos genéticos e programação genética} - Nelson Lago
  \item \textit{Programação dinâmica e Aprendizado por reforço aplicado ao jogo Tetris} - Suelen
    Carvalho
  \item Relatório final
\end{itemize}

\paragraph{
  Para esta semana planejou-se assistir as palestras dos alunos de mestrado em MAC0425 Inteligência
  Artificial, em especial as palestras citadas acima. Além disso essa semana foi reservada para o
  relatório final da matéria de MAC0215.
}

\section{Resultados}

\paragraph{
  Nesta seção irei primeiro descrever os resultados pessoais desta matéria e do projeto e em
  seguida irei abordar os resultados acadêmicos do projeto.
}

\paragraph{
  Durante o semestre aprendi muito sobre Inteligência Artificial. É uma área que pretendo seguir no
  futuro, especialmente com relação a pesquisa, e gostei bastante do assunto que estou estudando.
  Por isso, acho que meus estudos em MAC0215 impactaram fortemente na minha produtividade da
  Iniciação Científica. Com \textit{deadlines} impostos, me senti pressionado a estudar e a
  apresentar resultados. Não conseguiria apresentar os resultados de forma tão produtiva se não
  tivesse me inscrito nesta matéria.
}

\paragraph{
  Como a área que estou estudando -- principalmente neste estágio dos meus estudos -- é muito
  teórica, apresentar resultados concretos é difícil. Meus resultados neste projeto foram mais em
  adquirir conhecimento e construir uma base para que, em seguida, consiga atingir resultados mais
  concretos. Além disso, por essa área em especial ser relativamente nova (o primeiro paper em SPNs
  é de 2011, enquanto que a primeira vaga ideia de um circuito aritmético similar a SPNs data de
  2003\cite{diff-approach-darwiche}), o conteúdo base por trás de toda esta teoria é relativamente
  grande, já que se trata de tecnologia de ponta. Portanto, os únicos resultados que posso
  apresentar são os relatórios que escrevi, os quais planejo continuar escrevendo mesmo após a
  conclusão da matéria, já que são um jeito muito bom de tanto aprender a matéria quanto arquivar
  para futuras consultas.
}

\section{Conclusão}

\paragraph{
  Cada semana foi dedicado sábado e domingo para se escrever os relatórios, enquanto que de segunda
  a sexta eram lidos os artigos ou capítulos do livro. No entanto, em casos em que houvesse EPs ou
  provas, foi reservada a semana para estudo ou para os EPs. Por causa disso existem algumas
  semanas que não possuem relatório. Em casos em que a matéria lida era muito grande e não era
  possível escrever o relatório a tempo, juntou-se as duas semanas em um relatório.
}

\paragraph{
  Apesar de tudo isso, acredito que estudei no mínimo 100 horas. Os relatórios ajudam tanto a
  memorizar quando a revisar o conteúdo visto durante a semana, e portanto acredito serem uma forma
  de estudo.
}

\paragraph{
  Adicionalmente, as aulas de MAC0425 (Inteligência Artificial) contribuíram bastante no estudo da
  Iniciação Científica, já que o conteúdo de aula e o que foi visto no projeto tem partes iguais,
  principalmente os assuntos sobre Aprendizado e Incerteza na aula de MAC0425 são diretamente
  relacionados ao projeto.
}

\newpage

\printbibliography

\end{document}
